\section{Semaine 7 - 25 mars}

\subsection{Lundi}

Aujord'hui j'ai travaillé sur le code de Victor. En gros il a
essayé de faire une fonction \pythoninline{fit} qui généralise
les fonctions de Quentin.
\begin{python}
def fit(fct,xdata,ydata,pas):
    j=np.size(xdata);A=np.zeros(j);f=np.zeros(j);phi=np.zeros(j)
    for i in range(j-pas-1):
        P=sc.optimize.curve_fit(fct, xdata[i:pas+i], ydata[i:pas+i])[0]
        A[i]=P[0]
        f[i]=ydata[i+int(pas/2)]
        phi[i]=P[1]
    return A,f,phi
\end{python}
Cependant elle ne marche comme elle devrait. Après beaucoup
d'effort nous y sommes arrivés:
\begin{python}
def fit(fct, xdata, ydata, interval_size, P0, step_size, maxfev=10000):

    N = np.size(xdata)
    new_xvalues = np.zeros(N)
    parameters = np.zeros((N, len(P0)))
    count = 0

    for i in range(0, N-interval_size-1, step_size):
        P = sc.optimize.curve_fit(fct, xdata[i:i+interval_size], ydata[i:i+interval_size], P0, maxfev)[0]
        if P[0] > 0.01 :
            parameters[count] = P
            new_xvalues[count] = xdata[i+int(interval_size/2)]
            count += 1
            P0 = P
    print(f'\nCurve fitted, input size is {N}, output size is {count}')
return  new_xvalues[:count], parameters[:count].T
\end{python}

Avec cette fonction on a pu fitter les envelopes pour la data de l'air et du silicium.

\section{External listing highlighting}

%\pythonexternal{}

\section{Inline highlighting}

Definition \pythoninline{class MyClass} means \dots

\subsection{Mardi}

